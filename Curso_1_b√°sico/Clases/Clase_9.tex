\documentclass[11pt]{beamer}
\usepackage[utf8]{inputenc}
\usepackage[T1]{fontenc}
\usepackage{lmodern}
\usepackage{graphicx}
\usepackage{lipsum}
\usepackage{ragged2e}
\usepackage[spanish]{babel}
\usepackage{amsfonts}

\usepackage{authblk} 




\usetheme{Madrid}




%\definecolor{persianindigo}{rgb}{0.2, 0.07, 0.48}
%\definecolor{blue(pigment)}{rgb}{0.2, 0.2, 0.6}
%\definecolor{cornflowerblue}{rgb}{0.39, 0.58, 0.93}
%\definecolor{charcoal}{rgb}{0.21, 0.27, 0.31}
%\definecolor{bittersweet}{rgb}{1.0, 0.44, 0.37}
%\definecolor{onyx}{rgb}{0.06, 0.06, 0.06}
%\definecolor{caribbeangreen}{rgb}{0.0, 0.8, 0.6}
%\definecolor{trueblue}{rgb}{0.0,0.45,0.81}
\definecolor{alizarin}{rgb}{0.82, 0.1, 0.26}
\usecolortheme[named=alizarin]{structure}









\begin{document}
\justifying
\author{Juan Carlos Tapia}
\title{Mi primer documento en \LaTeX}
%\subtitle{}
% \logo{\includegraphics[width=1.5cm]{logik}}
\institute[UNPRG]{Universidad Nacional Pedro Ruiz Gallo}
%\date{}
%\subject{}
%\setbeamercovered{transparent}
%\setbeamertemplate{navigation symbols}{}



\begin{frame}[plain]
\maketitle
\end{frame}



\begin{frame}{Contenido}
\tableofcontents
\end{frame}





\section{Velos}

\begin{frame}{Título de la página}
\framesubtitle{Subtítulo}
\lipsum[1]
\end{frame}


\begin{frame}{Velos}
Texto
\begin{enumerate}[<+->]
\item Libiton
\item Justin
\item Luis
\item Anderson
\end{enumerate}
\end{frame}

\begin{frame}{Velos}  %<i->
Texto
\begin{enumerate}
\item<1-> Uno
\item<2-> Dos
\item<4-> Cuatro
\item<3-> Tres
\item<2-> Dos
\end{enumerate}
\end{frame}
\section{Comando pause}

\begin{frame}{Comando pause}

Texto 1\\ \pause
Texto 2 \\ \pause
Texto 3\\
Texto 4 \\
\end{frame}


\section{Cajas}

\begin{frame}{Blocks}
Texto
\begin{block}{Título}
\begin{center}
Esta es una caja
\end{center}
\end{block}
\end{frame}

\section{Ecuaciones}

\begin{frame}{Ecuaciones}
\begin{block}{Ecuación}
$$a^2 +b^2 =c^2$$
\end{block}
\end{frame}

\section{figuras}

\begin{frame}{Figuras}
   % \begin{center}
   %  \includegraphics[width=5cm]{logik}
   % \end{center}
\end{frame}

\section{Tablas}
\begin{frame}{Tablas}
\begin{table}[t]
\begin{center}
\begin{tabular}{| r | l | c |}
\hline
Fruta & Cantidad & Origen \\ \hline
Manzana & 4 & Estados Unidos \\
Naranja & 10 & España \\
Plátano & 3 & Colombia \\ \hline
\end{tabular}
\caption{Fruta disponible}
\label{tab:fruta}
\end{center}
\end{table}
\end{frame}


\section{Liga}
% \begin{frame}{Ligas}
% \href{https://manualdelatex.com/}{\includegraphics[width=3cm]{LATEX}}
% \end{frame}

\end{document}