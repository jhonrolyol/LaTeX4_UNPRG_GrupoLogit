\documentclass[11pt,a4paper]{article}
\usepackage[utf8]{inputenc}
\usepackage[T1]{fontenc}
\usepackage[spanish,es-tabla]{babel}
\usepackage{amsmath}
\usepackage{amsfonts}
\usepackage{amssymb}
\usepackage{graphicx}
\usepackage{caption}
\usepackage[left=2.54cm, right=2.54cm, top=2.54cm, bottom=2.54cm]{geometry}
\author{Juan Carlos Tapia}
\title{Figuras}
\begin{document}
\maketitle
\tableofcontents
\listoffigures
\listoftables

\section{width}
\subsection{Centrado}
\begin{center}
\includegraphics[width=5cm]{logik}
\end{center}

\subsection{A la izquierda}
\begin{flushleft}
\includegraphics[width=5cm]{logik}
\end{flushleft}


\subsection{A la Derecha}
\begin{flushright}
\includegraphics[width=5cm]{logik}
\end{flushright}


\section{height}
\subsection{Centrado}
\begin{center}
\includegraphics[height=5cm]{logik}
\end{center}


\subsection{A la izquierda}
\begin{flushleft}
\includegraphics[height=5cm]{logik}
\end{flushleft}

\subsection{A la Derecha}
\begin{flushright}
\includegraphics[height=5cm]{logik}
\end{flushright}
\section{height}

\section{scale}
\subsection{Centrado}
\begin{center}
\includegraphics[scale=0.35]{logik}
\end{center}


\subsection{A la izquierda}
\begin{flushleft}
\includegraphics[scale=0.35]{logik}
\end{flushleft}


\subsection{A la izquierda}
\begin{flushright}
\includegraphics[scale=0.35]{logik}
\end{flushright}

\section{Título de la imagen}

\begin{figure}[h]
\begin{center}
\includegraphics[width=7cm]{logik}
\caption{Logo de {\bf Logik}}
\end{center}
\end{figure}

\begin{tabular}{c|c|c}
\hline 
Imagen dentro de una tabla & \includegraphics[width=2cm]{logik} & \includegraphics[width=2cm]{UNPRG}\\ [0.5cm]\hline
\end{tabular}

\begin{figure}[h!]
\begin{center}
\begin{tabular}{|c|c|}
\hline \\ [0.2cm]
\includegraphics[width=2cm]{logik} & \includegraphics[width=2cm]{UNPRG} \\  \hline 
\end{tabular}
\caption{Ejemplo}
\end{center}
\end{figure}

\section{Rotación}

\begin{figure}[h!]
\begin{center}
\includegraphics[width=6cm,angle=-45 ]{logik}\\
\caption{Giro de -45°}
\includegraphics[width=6cm,angle=45 ]{logik}\\
\caption{Giro de 45°}
\includegraphics[width=6cm,angle=90 ]{logik}\\
\caption{Giro de 90°}
\includegraphics[width=6cm,angle=180 ]{logik}\\
\caption{Giro de 180°}
\includegraphics[width=6cm,angle=270 ]{logik}\\
\caption{Giro de 270°}
\end{center}
\end{figure}




\end{document}