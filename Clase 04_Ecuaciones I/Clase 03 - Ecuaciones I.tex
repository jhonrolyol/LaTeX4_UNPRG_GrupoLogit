\documentclass[11pt,a4paper]{article}
\usepackage[utf8]{inputenc}
\usepackage[T1]{fontenc}
\usepackage[spanish]{babel}
\usepackage{amsmath}
\usepackage{amsfonts}
\usepackage{amssymb}
\usepackage{graphicx}
\usepackage[left=2.54cm, right=2.54cm, top=2.54cm, bottom=2.54cm]{geometry}
\author{Juan Carlos Tapia}
\title{Ecuaciones I}
\begin{document}
\maketitle
Esta es mi primera ecuación:
\section{Potencias, subíndices y superbíndices}
$x^n$ , $2^5$, $x^{n+1}$, $(2^2)^2$, $2^{3^4}$\\
$x_n$, $2_5$, $x_{n+1}$, $(x_n)^2$\\
$x_i^2$, $$\int_a^b$$

\section{Fracciones}
Esta es una fracción:

Esta es una fracción tamaño texto:
$\tfrac{x+1}{x-1}$ del modo matemático\\
 
Esta es una fracción tamaño normal:
$\dfrac{x+1}{x-1}$ del modo matemático\\
 
Esta es un nueva fracción: $\left(\dfrac{1+\frac{2}{5}}{3-\frac{1}{6}}\right)$ en modo matemático.
 
Esta es una fracción tamaño normal:
$\displaystyle\frac{x+1}{x-1}$ del modo matemático\\
 
x normal, $x$ modo matemático\\
 
$$\frac{x+1}{x-1}$$
 
\section{Integrales}
Esta es un integral dentro de un texto $\displaystyle\int_{a}^{b}f(x)dx$ en modo matemático.
$$\int_a^bf(x)\,dx$$
$$\int f(x)dx$$
$$\int_{a}^{b}\int_{c}^{d}f(x,y)$$
 
Sea la función $f$ definida por: $f(x)=x^{x+1}$
$$\int_{a}^{b} f(x)dx$$
$$\int_{a}^{b} \left( x^{x+x^{x+1}}\right)dx$$
 
\section{Raíces}
 
$$\sqrt{x}$$
$$\sqrt[3]{x+1}$$
$$\sqrt[n]{\frac{x+1}{x-1}}$$
$$\frac{\sqrt{x}}{\sqrt[3]{x^2}}$$
 
\section{Delimitadores}
$$\left[\frac{x+1}{x-1}\right]^2$$
$$\left(\frac{x+1}{x-1}\right)^2$$
 
\section{Letras griegas}
$\alpha,\beta, \gamma, \delta,\epsilon,\theta,\pi,\sigma$
\end{document}