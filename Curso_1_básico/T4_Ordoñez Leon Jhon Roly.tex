


%%%%Preámbulo
% Tipo de documento
\documentclass[11pt]{article}



% Paquetes
\usepackage[utf8]{inputenc}
\usepackage[spanish]{babel}
\usepackage{latexsym,amsmath,amssymb,amsfonts,amsthm} 
\usepackage{graphicx}
\usepackage{ragged2e}
\usepackage{xcolor}
\usepackage{mathrsfs}
\usepackage{cancel}

% Colores
\definecolor{verdeclaro}{rgb}{0.0, 1.0, 0.5}
\definecolor{rojo}{rgb}{0.89, 0.04, 0.36}
\definecolor{violeta}{rgb}{0.89, 0.04, 0.36}
\definecolor{letra}{RGB}{13,17,14}
\definecolor{title}{rgb}{1.0,0.03,0.0}
\definecolor{pagecolor}{RGB}{13,17,14}


% Título
\title{Ecuaciones I}
\author{Jhon Roly Ordoñez Leon}
\date{\today}


%%% Documento
\begin{document}
    \maketitle
    \section{Ecuaciones I}
     \begin{itemize}
         \item[a)] Ecuación 1
         \[
            \left[ \dfrac{x + y^{n+1}}{m^{x+\frac{1}{2+1}}}  \right] +  \left( y- \dfrac{1}{\sqrt{2}} + \left(\dfrac{1}{x} \right)^{n+2}  \right)^{y-1}
         \]
         \item[b)] Ecuación 2
         \[
             \dfrac{ \dfrac{(x+\frac{5}{7})^{\sqrt[5]{x}}}{\sqrt{\frac{1}{2}+(x^{2})^{\frac{1}{5}}}} +x^{\left(\dfrac{1}{x-1} + \dfrac{\sqrt{x}^{7}}{5} \right)}   }{x-\dfrac{1+ \left(\frac{1}{\sqrt{3}}\right)  }{x-\sqrt{\frac{1}{5}}} }
         \]
         \item[c)] Integral 1
         \[
             \int\dfrac{x+1}{2}dx
         \]
         \item[d)] Integral 2
         \[
             \int\limits_{4}^{10} \dfrac{x^{2}+5}{5^{\frac{1}{5}}}dx
         \]
         \item[e)] Integral 3
         \[
            \int\limits_{4}^{10} \dfrac{  \frac{x^{2}}{1+\frac{1}{x}} +  \left(\sqrt[3]{x^{5}} \right)^{x+1} }{5^{\frac{1}{x}} + \left(\dfrac{x-1}{2+x^{5}}\right)^{5} }dx
         \] 
     \end{itemize}
     \newpage
      \section{Las 17 ecuaciones que cambaron el mundo}
      \begin{itemize}
        \item[1)] El teorema de Pitagoras (Pitagoras,530 a.C.)
        \[
            a^{2}+b^{2} = c^{2} 
        \]
        \item[2)] Logaritmos ( John Napier, 1610)
        \[
            \log{(xy)} = \log{(x)} + \log{(y)}
        \] 
        \item[3)] Cálculo ( Newton, 1668)
        \[
           \dfrac{df}{dx} = \lim\limits_{h \to 0} \dfrac{f(t+h)-f(t)}{h}
        \]
        \item[4)] Ley de la gravedad (Newton, 1687)
        \[
           F = G \dfrac{m_{1}m_{2}}{d^{2}} 
        \]
        \item[5)] Raiz cuadrada de menos uno (Euler, 1750)
        \[
            i^{2} = -1
        \]
        \item[6)] La fórmula de Euler para poliedros (Euler, 1751)
        \[
            V - E + F = 2
        \]
        \item[7)] Distribución Normal ( C.F. Gauss, 1810)
        \[
            \phi{(x)} = \dfrac{1}{\sigma\sqrt{2p}}\exp{\left\{-\dfrac{(x-u)^{2}}{2\sigma^{2}}\right\}}
        \]
        \item[8)] Ecuación de Onda ( J. dÁmbert, 1746)
        \[
            \dfrac{\alpha^{2}u}{\alpha t^{2}} = c^{2} \dfrac{\alpha^{2}u}{\alpha x^{2}}
        \]
        \item[9)] Transformada de Fourier ( j. Fourier, 1822)
        \[
            \hat{f}(\delta) = \int\limits_{-\infty}^{+\infty}f(x) e^{-2i\pi x \delta } dx 
        \]
        \item[10)] Ecuaciones de Navier-Stokes (C. Navier, G.Stokes, 1845)
        \[
            p\left(  \dfrac{\alpha v}{\alpha t} + v\cdot \nabla v \right) = -\nabla p + \nabla \cdot T + f
        \]
      \end{itemize}
      \newpage
      \section{Mis formulas favoritas}

      \begin{equation*}
          \textcolor{red}{\int\limits_{a}^{b}f(x)dx = \lim\limits_{n \to \infty} \sum\limits_{i=1}^{n}f(x_{i-1}) \Delta x}
      \end{equation*}

      \begin{equation*}
          \textcolor{blue}{\int\limits_{a}^{b}f(x)dx = F(x)\bigg|_{a}^{b} = F(b)-F(a)}
      \end{equation*}

      \begin{equation*}
          \textcolor{violeta}{\mathscr{L}{\{ f(t); t \to s\}} = \int\limits_{0}^{\infty}e^{-st}f(t)dt}
      \end{equation*}

      \begin{center}
          GRACIAS POR COMPARTIR CONOCIMIENTO 
      \end{center}
      \begin{center}
        DIOS TE BENDIGA !!!
    \end{center}

\end{document}







